\subsubsection{Tree Traversal}

\paragraph{
    There are three ways to traverse a binary search tree.\\
    In-order traversal, pre-order traversal, and post-order traversal.\\
    Inorder traversal visits nodes in a binary tree in the following order:\\
    Visit the left subtree.\\
    Visit the root node.\\
    Visit the right subtree.\\
    This traversal method is particularly useful for binary search trees (BSTs) because it visits the nodes in ascending order.\\
}

\begin{verbatim}
    Example:
    Consider the following binary search tree (BST):
       4
      / \
     2   6
    / \ / \
    1  3 5  7
    Inorder traversal of this tree would be: 1, 2, 3, 4, 5, 6, 7.

    Python Code:
    class Node:
        def __init__(self, key):
            self.left = None
            self.right = None
            self.val = key

    def inorder_traversal(root):
        if root:
            # Traverse the left subtree
            inorder_traversal(root.left)
            # Visit the root node
            print(root.val, end=' ')
            # Traverse the right subtree
            inorder_traversal(root.right)
\end{verbatim}

\paragraph{
    Preorder traversal visits nodes in the following order:\\
    Visit the root node.\\
    Visit the left subtree.\\
    Visit the right subtree.\\
    Preorder traversal is useful for creating a copy of the tree or getting a prefix expression of an expression tree.\\
}

\begin{verbatim}
    Example:
    Consider the same BST:
       4
      / \
     2   6
    / \ / \
    1  3 5  7
    Preorder traversal of this tree would be: 4, 2, 1, 3, 6, 5, 7.

    Python Code:
    def preorder_traversal(root):
        if root:
            # Visit the root node
            print(root.val, end=' ')
            # Traverse the left subtree
            preorder_traversal(root.left)
            # Traverse the right subtree
            preorder_traversal(root.right)
\end{verbatim}

\paragraph{
    Postorder traversal visits nodes in the following order:\\
    Visit the left subtree.\\
    Visit the right subtree.\\
    Visit the root node.\\
    Postorder traversal is useful for deleting a tree or evaluating postfix expressions of an expression tree.\\
}

\begin{verbatim}
    Example:
    Consider the same BST:
       4
      / \
     2   6
    / \ / \
    1  3 5  7
    Postorder traversal of this tree would be: 1, 3, 2, 5, 7, 6, 4.

    Python Code:
    def postorder_traversal(root):
        if root:
            # Traverse the left subtree
            postorder_traversal(root.left)
            # Traverse the right subtree
            postorder_traversal(root.right)
            # Visit the root node
            print(root.val, end=' ')
\end{verbatim}