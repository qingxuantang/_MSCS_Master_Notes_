% Options for packages loaded elsewhere
\PassOptionsToPackage{unicode}{hyperref}
\PassOptionsToPackage{hyphens}{url}
%
\documentclass[
]{article}
\usepackage{amsmath,amssymb}
\usepackage{lmodern}
\usepackage{iftex}
\ifPDFTeX
  \usepackage[T1]{fontenc}
  \usepackage[utf8]{inputenc}
  \usepackage{textcomp} % provide euro and other symbols
\else % if luatex or xetex
  \usepackage{unicode-math}
  \defaultfontfeatures{Scale=MatchLowercase}
  \defaultfontfeatures[\rmfamily]{Ligatures=TeX,Scale=1}
\fi
% Use upquote if available, for straight quotes in verbatim environments
\IfFileExists{upquote.sty}{\usepackage{upquote}}{}
\IfFileExists{microtype.sty}{% use microtype if available
  \usepackage[]{microtype}
  \UseMicrotypeSet[protrusion]{basicmath} % disable protrusion for tt fonts
}{}
\makeatletter
\@ifundefined{KOMAClassName}{% if non-KOMA class
  \IfFileExists{parskip.sty}{%
    \usepackage{parskip}
  }{% else
    \setlength{\parindent}{0pt}
    \setlength{\parskip}{6pt plus 2pt minus 1pt}}
}{% if KOMA class
  \KOMAoptions{parskip=half}}
\makeatother
\usepackage{xcolor}
\usepackage{color}
\usepackage{fancyvrb}
\newcommand{\VerbBar}{|}
\newcommand{\VERB}{\Verb[commandchars=\\\{\}]}
\DefineVerbatimEnvironment{Highlighting}{Verbatim}{commandchars=\\\{\}}
% Add ',fontsize=\small' for more characters per line
\newenvironment{Shaded}{}{}
\newcommand{\AlertTok}[1]{\textcolor[rgb]{1.00,0.00,0.00}{\textbf{#1}}}
\newcommand{\AnnotationTok}[1]{\textcolor[rgb]{0.38,0.63,0.69}{\textbf{\textit{#1}}}}
\newcommand{\AttributeTok}[1]{\textcolor[rgb]{0.49,0.56,0.16}{#1}}
\newcommand{\BaseNTok}[1]{\textcolor[rgb]{0.25,0.63,0.44}{#1}}
\newcommand{\BuiltInTok}[1]{\textcolor[rgb]{0.00,0.50,0.00}{#1}}
\newcommand{\CharTok}[1]{\textcolor[rgb]{0.25,0.44,0.63}{#1}}
\newcommand{\CommentTok}[1]{\textcolor[rgb]{0.38,0.63,0.69}{\textit{#1}}}
\newcommand{\CommentVarTok}[1]{\textcolor[rgb]{0.38,0.63,0.69}{\textbf{\textit{#1}}}}
\newcommand{\ConstantTok}[1]{\textcolor[rgb]{0.53,0.00,0.00}{#1}}
\newcommand{\ControlFlowTok}[1]{\textcolor[rgb]{0.00,0.44,0.13}{\textbf{#1}}}
\newcommand{\DataTypeTok}[1]{\textcolor[rgb]{0.56,0.13,0.00}{#1}}
\newcommand{\DecValTok}[1]{\textcolor[rgb]{0.25,0.63,0.44}{#1}}
\newcommand{\DocumentationTok}[1]{\textcolor[rgb]{0.73,0.13,0.13}{\textit{#1}}}
\newcommand{\ErrorTok}[1]{\textcolor[rgb]{1.00,0.00,0.00}{\textbf{#1}}}
\newcommand{\ExtensionTok}[1]{#1}
\newcommand{\FloatTok}[1]{\textcolor[rgb]{0.25,0.63,0.44}{#1}}
\newcommand{\FunctionTok}[1]{\textcolor[rgb]{0.02,0.16,0.49}{#1}}
\newcommand{\ImportTok}[1]{\textcolor[rgb]{0.00,0.50,0.00}{\textbf{#1}}}
\newcommand{\InformationTok}[1]{\textcolor[rgb]{0.38,0.63,0.69}{\textbf{\textit{#1}}}}
\newcommand{\KeywordTok}[1]{\textcolor[rgb]{0.00,0.44,0.13}{\textbf{#1}}}
\newcommand{\NormalTok}[1]{#1}
\newcommand{\OperatorTok}[1]{\textcolor[rgb]{0.40,0.40,0.40}{#1}}
\newcommand{\OtherTok}[1]{\textcolor[rgb]{0.00,0.44,0.13}{#1}}
\newcommand{\PreprocessorTok}[1]{\textcolor[rgb]{0.74,0.48,0.00}{#1}}
\newcommand{\RegionMarkerTok}[1]{#1}
\newcommand{\SpecialCharTok}[1]{\textcolor[rgb]{0.25,0.44,0.63}{#1}}
\newcommand{\SpecialStringTok}[1]{\textcolor[rgb]{0.73,0.40,0.53}{#1}}
\newcommand{\StringTok}[1]{\textcolor[rgb]{0.25,0.44,0.63}{#1}}
\newcommand{\VariableTok}[1]{\textcolor[rgb]{0.10,0.09,0.49}{#1}}
\newcommand{\VerbatimStringTok}[1]{\textcolor[rgb]{0.25,0.44,0.63}{#1}}
\newcommand{\WarningTok}[1]{\textcolor[rgb]{0.38,0.63,0.69}{\textbf{\textit{#1}}}}
\setlength{\emergencystretch}{3em} % prevent overfull lines
\providecommand{\tightlist}{%
  \setlength{\itemsep}{0pt}\setlength{\parskip}{0pt}}
\setcounter{secnumdepth}{-\maxdimen} % remove section numbering
\ifLuaTeX
  \usepackage{selnolig}  % disable illegal ligatures
\fi
\IfFileExists{bookmark.sty}{\usepackage{bookmark}}{\usepackage{hyperref}}
\IfFileExists{xurl.sty}{\usepackage{xurl}}{} % add URL line breaks if available
\urlstyle{same} % disable monospaced font for URLs
\hypersetup{
  hidelinks,
  pdfcreator={LaTeX via pandoc}}

\author{}
\date{}

\begin{document}

\hypertarget{assignment-3}{%
\section{Assignment 3}\label{assignment-3}}

\hypertarget{problem-1-design-a-correct-partition-algorithm}{%
\subsection{Problem 1: Design a Correct Partition
Algorithm}\label{problem-1-design-a-correct-partition-algorithm}}

You are given code below for an incorrect partition algorithm that fails
to partition arrays wrongly or cause out of bounds access in arrays. The
comments include the invariants the algorithm wishes to maintain and
will help you debug.

Your goal is to write test cases that demonstrate that the partitioning
will fail in various ways.

\begin{Shaded}
\begin{Highlighting}[]
\KeywordTok{def}\NormalTok{ swap(a, i, j):}
    \ControlFlowTok{assert} \DecValTok{0} \OperatorTok{\textless{}=}\NormalTok{ i }\OperatorTok{\textless{}} \BuiltInTok{len}\NormalTok{(a), }\SpecialStringTok{f\textquotesingle{}accessing index }\SpecialCharTok{\{}\NormalTok{i}\SpecialCharTok{\}}\SpecialStringTok{ beyond end of array }\SpecialCharTok{\{}\BuiltInTok{len}\NormalTok{(a)}\SpecialCharTok{\}}\SpecialStringTok{\textquotesingle{}}
    \ControlFlowTok{assert} \DecValTok{0} \OperatorTok{\textless{}=}\NormalTok{ j }\OperatorTok{\textless{}} \BuiltInTok{len}\NormalTok{(a), }\SpecialStringTok{f\textquotesingle{}accessing index }\SpecialCharTok{\{}\NormalTok{j}\SpecialCharTok{\}}\SpecialStringTok{ beyond end of array }\SpecialCharTok{\{}\BuiltInTok{len}\NormalTok{(a)}\SpecialCharTok{\}}\SpecialStringTok{\textquotesingle{}}
\NormalTok{    a[i], a[j] }\OperatorTok{=}\NormalTok{ a[j], a[i]}

\KeywordTok{def}\NormalTok{ tryPartition(a):}
    \CommentTok{\# implementation of Lomuto partitioning algorithm}
\NormalTok{    n }\OperatorTok{=} \BuiltInTok{len}\NormalTok{(a)}
\NormalTok{    pivot }\OperatorTok{=}\NormalTok{ a[n}\OperatorTok{{-}}\DecValTok{1}\NormalTok{] }\CommentTok{\# choose last element as the pivot.}
\NormalTok{    i,j }\OperatorTok{=} \DecValTok{0}\NormalTok{,}\DecValTok{0} \CommentTok{\# initialize i and j both to be 0}
    \ControlFlowTok{for}\NormalTok{ j }\KeywordTok{in} \BuiltInTok{range}\NormalTok{(n}\OperatorTok{{-}}\DecValTok{1}\NormalTok{): }\CommentTok{\# j = 0 to n{-}2 (inclusive)}
        \CommentTok{\# Invariant: a[0] .. a[i] are \textless{}= pivot}
        \CommentTok{\#            a[i+1]...a[j{-}1] are \textgreater{} pivot}
        \ControlFlowTok{if}\NormalTok{ a[j] }\OperatorTok{\textless{}=}\NormalTok{ pivot:}
\NormalTok{            swap(a, i}\OperatorTok{+}\DecValTok{1}\NormalTok{, j)}
\NormalTok{            i }\OperatorTok{=}\NormalTok{ i }\OperatorTok{+} \DecValTok{1}
\NormalTok{    swap(a, i}\OperatorTok{+}\DecValTok{1}\NormalTok{, n}\OperatorTok{{-}}\DecValTok{1}\NormalTok{) }\CommentTok{\# place pivot in its correct place.}
    \ControlFlowTok{return}\NormalTok{ i}\OperatorTok{+}\DecValTok{1} \CommentTok{\# return the index where we placed the pivot}
\end{Highlighting}
\end{Shaded}

First write a function that will return True if an array is correctly
partitioned at index \texttt{k}. I.e, all elements at indices
\texttt{\textless{}\ k} are all \texttt{\textless{}=\ a{[}k{]}} and all
elements indices \texttt{\textgreater{}\ k} are all
\texttt{\textgreater{}\ a{[}k{]}}

\begin{Shaded}
\begin{Highlighting}[]
\KeywordTok{def}\NormalTok{ testIfPartitioned(a, k):}
    \CommentTok{\# }\AlertTok{TODO}\CommentTok{ : test if all elements at indices \textless{} k are all \textless{}= a[k]}
    \CommentTok{\#         and all elements at indices \textgreater{} k are all \textgreater{} a[k]}
    \CommentTok{\# return TRUE if the array is correctly partitioned around a[k] and return FALSE otherwise}
    \ControlFlowTok{assert} \DecValTok{0} \OperatorTok{\textless{}=}\NormalTok{ k }\OperatorTok{\textless{}} \BuiltInTok{len}\NormalTok{(a)}

    \CommentTok{\# your code here}
    \CommentTok{\# if k == 0:}
    \CommentTok{\#   left = []}
    \CommentTok{\#   right = a[k+1:]}
    \CommentTok{\#   for j in right:}
    \CommentTok{\#     if j \textless{} a[k]:}
    \CommentTok{\#       return False}
    \CommentTok{\# elif k == len(a){-}1:}
    \CommentTok{\#   left = a[:k]}
    \CommentTok{\#   right = []}
    \CommentTok{\#   for i in left:}
    \CommentTok{\#     if i \textgreater{} a[k]:}
    \CommentTok{\#       return False}
    \CommentTok{\# else:}
    \CommentTok{\#   left = a[:k]}
    \CommentTok{\#   right = a[k+1:]}
    \CommentTok{\#   for i,j in zip(left,right):}
    \CommentTok{\#     if i \textgreater{} a[k] or j \textless{} a[k]:}
    \CommentTok{\#       return False}

    \CommentTok{\# Check elements to the left of index k}
    \ControlFlowTok{for}\NormalTok{ i }\KeywordTok{in} \BuiltInTok{range}\NormalTok{(k):}
        \ControlFlowTok{if}\NormalTok{ a[i] }\OperatorTok{\textgreater{}}\NormalTok{ a[k]:}
            \ControlFlowTok{return} \VariableTok{False}

    \CommentTok{\# Check elements to the right of index k}
    \ControlFlowTok{for}\NormalTok{ j }\KeywordTok{in} \BuiltInTok{range}\NormalTok{(k}\OperatorTok{+}\DecValTok{1}\NormalTok{, }\BuiltInTok{len}\NormalTok{(a)):}
        \ControlFlowTok{if}\NormalTok{ a[j] }\OperatorTok{\textless{}=}\NormalTok{ a[k]:}
            \ControlFlowTok{return} \VariableTok{False}

    \ControlFlowTok{return} \VariableTok{True}
\end{Highlighting}
\end{Shaded}

\begin{Shaded}
\begin{Highlighting}[]
\ControlFlowTok{assert}\NormalTok{ testIfPartitioned([}\OperatorTok{{-}}\DecValTok{1}\NormalTok{, }\DecValTok{5}\NormalTok{, }\DecValTok{2}\NormalTok{, }\DecValTok{3}\NormalTok{, }\DecValTok{4}\NormalTok{, }\DecValTok{8}\NormalTok{, }\DecValTok{9}\NormalTok{, }\DecValTok{14}\NormalTok{, }\DecValTok{10}\NormalTok{, }\DecValTok{23}\NormalTok{],}\DecValTok{5}\NormalTok{) }\OperatorTok{==} \VariableTok{True}\NormalTok{, }\StringTok{\textquotesingle{} Test \# 1 failed.\textquotesingle{}}
\ControlFlowTok{assert}\NormalTok{ testIfPartitioned([}\OperatorTok{{-}}\DecValTok{1}\NormalTok{, }\DecValTok{5}\NormalTok{, }\DecValTok{2}\NormalTok{, }\DecValTok{3}\NormalTok{, }\DecValTok{4}\NormalTok{, }\DecValTok{8}\NormalTok{, }\DecValTok{9}\NormalTok{, }\DecValTok{14}\NormalTok{, }\DecValTok{11}\NormalTok{, }\DecValTok{23}\NormalTok{],}\DecValTok{4}\NormalTok{) }\OperatorTok{==} \VariableTok{False}\NormalTok{, }\StringTok{\textquotesingle{} Test \# 2 failed.\textquotesingle{}}
\ControlFlowTok{assert}\NormalTok{ testIfPartitioned([}\OperatorTok{{-}}\DecValTok{1}\NormalTok{, }\DecValTok{5}\NormalTok{, }\DecValTok{2}\NormalTok{, }\DecValTok{3}\NormalTok{, }\DecValTok{4}\NormalTok{, }\DecValTok{8}\NormalTok{, }\DecValTok{9}\NormalTok{, }\DecValTok{14}\NormalTok{, }\DecValTok{23}\NormalTok{, }\DecValTok{21}\NormalTok{],}\DecValTok{0}\NormalTok{) }\OperatorTok{==} \VariableTok{True}\NormalTok{, }\StringTok{\textquotesingle{} Test \# 3 failed.\textquotesingle{}}
\ControlFlowTok{assert}\NormalTok{ testIfPartitioned([}\OperatorTok{{-}}\DecValTok{1}\NormalTok{, }\DecValTok{5}\NormalTok{, }\DecValTok{2}\NormalTok{, }\DecValTok{3}\NormalTok{, }\DecValTok{4}\NormalTok{, }\DecValTok{8}\NormalTok{, }\DecValTok{9}\NormalTok{, }\DecValTok{14}\NormalTok{, }\DecValTok{22}\NormalTok{, }\DecValTok{23}\NormalTok{],}\DecValTok{9}\NormalTok{) }\OperatorTok{==} \VariableTok{True}\NormalTok{, }\StringTok{\textquotesingle{} Test \# 4 failed.\textquotesingle{}}
\ControlFlowTok{assert}\NormalTok{ testIfPartitioned([}\OperatorTok{{-}}\DecValTok{1}\NormalTok{, }\DecValTok{5}\NormalTok{, }\DecValTok{2}\NormalTok{, }\DecValTok{3}\NormalTok{, }\DecValTok{4}\NormalTok{, }\DecValTok{8}\NormalTok{, }\DecValTok{9}\NormalTok{, }\DecValTok{14}\NormalTok{, }\DecValTok{8}\NormalTok{, }\DecValTok{23}\NormalTok{],}\DecValTok{5}\NormalTok{) }\OperatorTok{==} \VariableTok{False}\NormalTok{, }\StringTok{\textquotesingle{} Test \# 5 failed.\textquotesingle{}}
\ControlFlowTok{assert}\NormalTok{ testIfPartitioned([}\OperatorTok{{-}}\DecValTok{1}\NormalTok{, }\DecValTok{5}\NormalTok{, }\DecValTok{2}\NormalTok{, }\DecValTok{3}\NormalTok{, }\DecValTok{4}\NormalTok{, }\DecValTok{8}\NormalTok{, }\DecValTok{9}\NormalTok{, }\DecValTok{13}\NormalTok{, }\DecValTok{9}\NormalTok{, }\OperatorTok{{-}}\DecValTok{11}\NormalTok{],}\DecValTok{5}\NormalTok{) }\OperatorTok{==} \VariableTok{False}\NormalTok{, }\StringTok{\textquotesingle{} Test \# 6 failed.\textquotesingle{}}
\ControlFlowTok{assert}\NormalTok{ testIfPartitioned([}\DecValTok{4}\NormalTok{, }\DecValTok{4}\NormalTok{, }\DecValTok{4}\NormalTok{, }\DecValTok{4}\NormalTok{, }\DecValTok{4}\NormalTok{, }\DecValTok{8}\NormalTok{, }\DecValTok{9}\NormalTok{, }\DecValTok{13}\NormalTok{, }\DecValTok{9}\NormalTok{, }\DecValTok{11}\NormalTok{],}\DecValTok{4}\NormalTok{) }\OperatorTok{==} \VariableTok{True}\NormalTok{, }\StringTok{\textquotesingle{} Test \# 7 failed.\textquotesingle{}}
\BuiltInTok{print}\NormalTok{(}\StringTok{\textquotesingle{}Passed all tests (10 points)\textquotesingle{}}\NormalTok{)}
\end{Highlighting}
\end{Shaded}

\begin{verbatim}
Passed all tests (10 points)
\end{verbatim}

\begin{Shaded}
\begin{Highlighting}[]
\CommentTok{\# Write an array called a1 that will be incorrectly partitioned by the tryPartition algorithm above}
\CommentTok{\# Your input when run on tryPartition algorithm should raise an out of bounds array access exception}
\CommentTok{\# in the swap function or fail to partition correctly.}

\CommentTok{\#\# Define an array a1 below of length \textgreater{} 0 that will be incorrectly partitioned by tryPartition algorithm.}
\CommentTok{\#\# We will test whether your solution works in the subsequent cells.}
\CommentTok{\# your code here}
\NormalTok{a1 }\OperatorTok{=}\NormalTok{ [}\OperatorTok{{-}}\DecValTok{1}\NormalTok{,}\DecValTok{4}\NormalTok{,}\DecValTok{8}\NormalTok{]}

\ControlFlowTok{assert}\NormalTok{( }\BuiltInTok{len}\NormalTok{(a1) }\OperatorTok{\textgreater{}} \DecValTok{0}\NormalTok{)}

\CommentTok{\# Write an array called a2 that will be incorrectly partitioned by the tryPartition algorithm above}
\CommentTok{\# Your input when run on tryPartition algorithm should raise an out of bounds array access exception}
\CommentTok{\# in the swap function or fail to partition correctly.}
\CommentTok{\# a2 must be different from a1}

\CommentTok{\# your code here}
\NormalTok{a2 }\OperatorTok{=}\NormalTok{ [}\DecValTok{9}\NormalTok{,}\DecValTok{2}\NormalTok{,}\DecValTok{0}\NormalTok{]}

\ControlFlowTok{assert}\NormalTok{( }\BuiltInTok{len}\NormalTok{(a2) }\OperatorTok{\textgreater{}} \DecValTok{0}\NormalTok{)}
\ControlFlowTok{assert}\NormalTok{ (a1 }\OperatorTok{!=}\NormalTok{ a2)}


\CommentTok{\# Write an array called a3 that will be incorrectly partitioned by the tryPartition algorithm above}
\CommentTok{\# Your input when run on tryPartition algorithm should raise an out of bounds array access exception}
\CommentTok{\# in the swap function or fail to partition correctly.}
\CommentTok{\# a3 must be different from a1, a2}

\CommentTok{\# your code here}
\NormalTok{a3 }\OperatorTok{=}\NormalTok{ [}\DecValTok{2}\NormalTok{,}\DecValTok{9}\NormalTok{,}\OperatorTok{{-}}\DecValTok{5}\NormalTok{,}\DecValTok{4}\NormalTok{,}\DecValTok{3}\NormalTok{]}

\ControlFlowTok{assert}\NormalTok{( }\BuiltInTok{len}\NormalTok{(a3) }\OperatorTok{\textgreater{}} \DecValTok{0}\NormalTok{)}
\ControlFlowTok{assert}\NormalTok{ (a3 }\OperatorTok{!=}\NormalTok{ a2)}
\ControlFlowTok{assert}\NormalTok{ (a3 }\OperatorTok{!=}\NormalTok{ a1)}

\KeywordTok{def}\NormalTok{ dummyFunction():}
    \ControlFlowTok{pass}
\end{Highlighting}
\end{Shaded}

\begin{Shaded}
\begin{Highlighting}[]
\CommentTok{\# your code here}
\end{Highlighting}
\end{Shaded}

\begin{Shaded}
\begin{Highlighting}[]
\ControlFlowTok{try}\NormalTok{:}
\NormalTok{    j1 }\OperatorTok{=}\NormalTok{ tryPartition(a1)}
    \ControlFlowTok{assert} \KeywordTok{not}\NormalTok{ testIfPartitioned(a1, j1)}
    \BuiltInTok{print}\NormalTok{(}\StringTok{\textquotesingle{}Partitioning was unsuccessful {-} this is what you were asked to break the code\textquotesingle{}}\NormalTok{)}
\ControlFlowTok{except} \PreprocessorTok{Exception} \ImportTok{as}\NormalTok{ e:}
    \BuiltInTok{print}\NormalTok{(}\SpecialStringTok{f\textquotesingle{}Assertion failed }\SpecialCharTok{\{}\NormalTok{e}\SpecialCharTok{\}}\SpecialStringTok{ {-} this is fine since you were asked to break the code.\textquotesingle{}}\NormalTok{)}

\ControlFlowTok{try}\NormalTok{:}
\NormalTok{    j2 }\OperatorTok{=}\NormalTok{ tryPartition(a2)}
    \ControlFlowTok{assert} \KeywordTok{not}\NormalTok{ testIfPartitioned(a2, j2)}
\ControlFlowTok{except} \PreprocessorTok{Exception} \ImportTok{as}\NormalTok{ e:}
    \BuiltInTok{print}\NormalTok{(}\SpecialStringTok{f\textquotesingle{}Assertion failed }\SpecialCharTok{\{}\NormalTok{e}\SpecialCharTok{\}}\SpecialStringTok{ {-} this is fine since you were asked to break the code.\textquotesingle{}}\NormalTok{)}


\ControlFlowTok{try}\NormalTok{:}
\NormalTok{    j3 }\OperatorTok{=}\NormalTok{ tryPartition(a3)}
    \ControlFlowTok{assert} \KeywordTok{not}\NormalTok{ testIfPartitioned(a3, j3)}
\ControlFlowTok{except} \PreprocessorTok{Exception} \ImportTok{as}\NormalTok{ e:}
    \BuiltInTok{print}\NormalTok{(}\SpecialStringTok{f\textquotesingle{}Assertion failed }\SpecialCharTok{\{}\NormalTok{e}\SpecialCharTok{\}}\SpecialStringTok{ {-} this is fine since you were asked to break the code.\textquotesingle{}}\NormalTok{)}

\NormalTok{dummyFunction()}

\BuiltInTok{print}\NormalTok{(}\StringTok{\textquotesingle{}Passed 5 points!\textquotesingle{}}\NormalTok{)}
\end{Highlighting}
\end{Shaded}

\begin{verbatim}
Assertion failed accessing index 3 beyond end of array 3 - this is fine since you were asked to break the code.
Assertion failed accessing index 5 beyond end of array 5 - this is fine since you were asked to break the code.
Passed 5 points!
\end{verbatim}

\hypertarget{debug-the-function}{%
\subsubsection{Debug the function}\label{debug-the-function}}

Point out where the bug is and what the fix is for the tryPartition
function. Note that the answer below is not graded.

YOUR ANSWER HERE

\hypertarget{problem-2-rapid-sorting-of-arrays-with-bounded-number-of-elements}{%
\subsection{Problem 2. Rapid Sorting of Arrays with Bounded Number of
Elements.}\label{problem-2-rapid-sorting-of-arrays-with-bounded-number-of-elements}}

Thus far, we have presented sorting algorithms that are
comparison-based. Ie., they make no assumptions about the elements in
the array just that we have a \texttt{\textless{}=} comparison operator.
We now ask you to develop a rapid sorting algorithm for an array of size
\(n\) when it is given to you that all elements in the array are between
\(1, \ldots, k\) for a given \(k\). Eg., consider an array with n =
100000 elements wherein all elements are between 1,..., k = 100.

Develop a sorting algorithm using partition that runs in
\(\Theta(n \times k)\) time for such arrays. \textbf{Hint} You can
choose your pivots in a simple manner each time.

\hypertarget{part-a}{%
\subsubsection{Part A}\label{part-a}}

Describe your algorithm as pseudocode and argue why it runs in time
\(\Theta(n \times k)\). This part will not be graded but is intended for
your own edification.

YOUR ANSWER HERE

\hypertarget{part-b}{%
\subsection{Part B}\label{part-b}}

Complete the implementation of a function \texttt{boundedSort(a,\ k)} by
completing the \texttt{simplePatition} function. Given an array
\texttt{a} and a fixed \texttt{pivot} element, it should partition the
array "in-place" so that all elements \texttt{\textless{}=\ pivot} are
on one side of the array and elements \texttt{\textgreater{}\ pivot} on
the other. You should not create a new array in your code.

\begin{Shaded}
\begin{Highlighting}[]

\KeywordTok{def}\NormalTok{ swap(a, i, j):}
    \ControlFlowTok{assert} \DecValTok{0} \OperatorTok{\textless{}=}\NormalTok{ i }\OperatorTok{\textless{}} \BuiltInTok{len}\NormalTok{(a), }\SpecialStringTok{f\textquotesingle{}accessing index }\SpecialCharTok{\{}\NormalTok{i}\SpecialCharTok{\}}\SpecialStringTok{ beyond end of array }\SpecialCharTok{\{}\BuiltInTok{len}\NormalTok{(a)}\SpecialCharTok{\}}\SpecialStringTok{\textquotesingle{}}
    \ControlFlowTok{assert} \DecValTok{0} \OperatorTok{\textless{}=}\NormalTok{ j }\OperatorTok{\textless{}} \BuiltInTok{len}\NormalTok{(a), }\SpecialStringTok{f\textquotesingle{}accessing index }\SpecialCharTok{\{}\NormalTok{j}\SpecialCharTok{\}}\SpecialStringTok{ beyond end of array }\SpecialCharTok{\{}\BuiltInTok{len}\NormalTok{(a)}\SpecialCharTok{\}}\SpecialStringTok{\textquotesingle{}}
\NormalTok{    a[i], a[j] }\OperatorTok{=}\NormalTok{ a[j], a[i]}



\KeywordTok{def}\NormalTok{ simplePartition(a, pivot):}
    \CommentTok{\#\# To do: partition the array a according to pivot.}
    \CommentTok{\# Your array must be partitioned into two regions {-} \textless{}= pivot followed by elements \textgreater{} pivot}
    \CommentTok{\#\# If an element at the beginning of the array is already \textless{}= pivot in the beginning of the array, it should not}
    \CommentTok{\#\# be moved by the algorithm.}
    \CommentTok{\# your code here}
\NormalTok{    left }\OperatorTok{=} \DecValTok{0}
\NormalTok{    right }\OperatorTok{=} \BuiltInTok{len}\NormalTok{(a) }\OperatorTok{{-}} \DecValTok{1}

    \ControlFlowTok{while}\NormalTok{ left }\OperatorTok{\textless{}=}\NormalTok{ right:}
      \ControlFlowTok{while}\NormalTok{ left }\OperatorTok{\textless{}=}\NormalTok{ right }\KeywordTok{and}\NormalTok{ a[left] }\OperatorTok{\textless{}=}\NormalTok{ pivot:}
\NormalTok{          left }\OperatorTok{+=} \DecValTok{1}
      \ControlFlowTok{while}\NormalTok{ left }\OperatorTok{\textless{}=}\NormalTok{ right }\KeywordTok{and}\NormalTok{ a[right] }\OperatorTok{\textgreater{}}\NormalTok{ pivot:}
\NormalTok{          right }\OperatorTok{{-}=} \DecValTok{1}
      \ControlFlowTok{if}\NormalTok{ left }\OperatorTok{\textless{}}\NormalTok{ right:}
\NormalTok{          swap(a, left, right)}
\NormalTok{          left }\OperatorTok{+=} \DecValTok{1}
\NormalTok{          right }\OperatorTok{{-}=} \DecValTok{1}







\KeywordTok{def}\NormalTok{ boundedSort(a, k):}
    \ControlFlowTok{for}\NormalTok{ j }\KeywordTok{in} \BuiltInTok{range}\NormalTok{(}\DecValTok{1}\NormalTok{, k):}
\NormalTok{        simplePartition(a, j)}
\end{Highlighting}
\end{Shaded}

\begin{Shaded}
\begin{Highlighting}[]
\NormalTok{a }\OperatorTok{=}\NormalTok{ [}\DecValTok{1}\NormalTok{, }\DecValTok{3}\NormalTok{, }\DecValTok{6}\NormalTok{, }\DecValTok{1}\NormalTok{, }\DecValTok{5}\NormalTok{, }\DecValTok{4}\NormalTok{, }\DecValTok{1}\NormalTok{, }\DecValTok{1}\NormalTok{, }\DecValTok{2}\NormalTok{, }\DecValTok{3}\NormalTok{, }\DecValTok{3}\NormalTok{, }\DecValTok{1}\NormalTok{, }\DecValTok{3}\NormalTok{, }\DecValTok{5}\NormalTok{, }\DecValTok{2}\NormalTok{, }\DecValTok{2}\NormalTok{, }\DecValTok{4}\NormalTok{]}
\BuiltInTok{print}\NormalTok{(a)}
\NormalTok{simplePartition(a, }\DecValTok{1}\NormalTok{)}
\BuiltInTok{print}\NormalTok{(a)}
\ControlFlowTok{assert}\NormalTok{(a[:}\DecValTok{5}\NormalTok{] }\OperatorTok{==}\NormalTok{ [}\DecValTok{1}\NormalTok{,}\DecValTok{1}\NormalTok{,}\DecValTok{1}\NormalTok{,}\DecValTok{1}\NormalTok{,}\DecValTok{1}\NormalTok{]), }\StringTok{\textquotesingle{}Simple partition test 1 failed\textquotesingle{}}

\NormalTok{simplePartition(a, }\DecValTok{2}\NormalTok{)}
\BuiltInTok{print}\NormalTok{(a)}
\ControlFlowTok{assert}\NormalTok{(a[:}\DecValTok{5}\NormalTok{] }\OperatorTok{==}\NormalTok{ [}\DecValTok{1}\NormalTok{,}\DecValTok{1}\NormalTok{,}\DecValTok{1}\NormalTok{,}\DecValTok{1}\NormalTok{,}\DecValTok{1}\NormalTok{]), }\StringTok{\textquotesingle{}Simple partition test 2(A) failed\textquotesingle{}}
\ControlFlowTok{assert}\NormalTok{(a[}\DecValTok{5}\NormalTok{:}\DecValTok{8}\NormalTok{] }\OperatorTok{==}\NormalTok{ [}\DecValTok{2}\NormalTok{,}\DecValTok{2}\NormalTok{,}\DecValTok{2}\NormalTok{]), }\StringTok{\textquotesingle{}Simple Partition test 2(B) failed\textquotesingle{}}


\NormalTok{simplePartition(a, }\DecValTok{3}\NormalTok{)}
\BuiltInTok{print}\NormalTok{(a)}
\ControlFlowTok{assert}\NormalTok{(a[:}\DecValTok{5}\NormalTok{] }\OperatorTok{==}\NormalTok{ [}\DecValTok{1}\NormalTok{,}\DecValTok{1}\NormalTok{,}\DecValTok{1}\NormalTok{,}\DecValTok{1}\NormalTok{,}\DecValTok{1}\NormalTok{]), }\StringTok{\textquotesingle{}Simple partition test 3(A) failed\textquotesingle{}}
\ControlFlowTok{assert}\NormalTok{(a[}\DecValTok{5}\NormalTok{:}\DecValTok{8}\NormalTok{] }\OperatorTok{==}\NormalTok{ [}\DecValTok{2}\NormalTok{,}\DecValTok{2}\NormalTok{,}\DecValTok{2}\NormalTok{]), }\StringTok{\textquotesingle{}Simple Partition test 3(B) failed\textquotesingle{}}
\ControlFlowTok{assert}\NormalTok{(a[}\DecValTok{8}\NormalTok{:}\DecValTok{12}\NormalTok{] }\OperatorTok{==}\NormalTok{ [}\DecValTok{3}\NormalTok{,}\DecValTok{3}\NormalTok{,}\DecValTok{3}\NormalTok{,}\DecValTok{3}\NormalTok{]), }\StringTok{\textquotesingle{}Simple Partition test 3(C) failed\textquotesingle{}}

\NormalTok{simplePartition(a, }\DecValTok{4}\NormalTok{)}
\BuiltInTok{print}\NormalTok{(a)}
\ControlFlowTok{assert}\NormalTok{(a[:}\DecValTok{5}\NormalTok{] }\OperatorTok{==}\NormalTok{ [}\DecValTok{1}\NormalTok{,}\DecValTok{1}\NormalTok{,}\DecValTok{1}\NormalTok{,}\DecValTok{1}\NormalTok{,}\DecValTok{1}\NormalTok{]), }\StringTok{\textquotesingle{}Simple partition test 4(A) failed\textquotesingle{}}
\ControlFlowTok{assert}\NormalTok{(a[}\DecValTok{5}\NormalTok{:}\DecValTok{8}\NormalTok{] }\OperatorTok{==}\NormalTok{ [}\DecValTok{2}\NormalTok{,}\DecValTok{2}\NormalTok{,}\DecValTok{2}\NormalTok{]), }\StringTok{\textquotesingle{}Simple Partition test 4(B) failed\textquotesingle{}}
\ControlFlowTok{assert}\NormalTok{(a[}\DecValTok{8}\NormalTok{:}\DecValTok{12}\NormalTok{] }\OperatorTok{==}\NormalTok{ [}\DecValTok{3}\NormalTok{,}\DecValTok{3}\NormalTok{,}\DecValTok{3}\NormalTok{,}\DecValTok{3}\NormalTok{]), }\StringTok{\textquotesingle{}Simple Partition test 4(C) failed\textquotesingle{}}
\ControlFlowTok{assert}\NormalTok{(a[}\DecValTok{12}\NormalTok{:}\DecValTok{14}\NormalTok{]}\OperatorTok{==}\NormalTok{[}\DecValTok{4}\NormalTok{,}\DecValTok{4}\NormalTok{]), }\StringTok{\textquotesingle{}Simple Partition test 4(D) failed\textquotesingle{}}

\NormalTok{simplePartition(a, }\DecValTok{5}\NormalTok{)}
\BuiltInTok{print}\NormalTok{(a)}
\ControlFlowTok{assert}\NormalTok{(a }\OperatorTok{==}\NormalTok{ [}\DecValTok{1}\NormalTok{]}\OperatorTok{*}\DecValTok{5}\OperatorTok{+}\NormalTok{[}\DecValTok{2}\NormalTok{]}\OperatorTok{*}\DecValTok{3}\OperatorTok{+}\NormalTok{[}\DecValTok{3}\NormalTok{]}\OperatorTok{*}\DecValTok{4}\OperatorTok{+}\NormalTok{[}\DecValTok{4}\NormalTok{]}\OperatorTok{*}\DecValTok{2}\OperatorTok{+}\NormalTok{[}\DecValTok{5}\NormalTok{]}\OperatorTok{*}\DecValTok{2}\OperatorTok{+}\NormalTok{[}\DecValTok{6}\NormalTok{]), }\StringTok{\textquotesingle{}Simple Parition test 5 failed\textquotesingle{}}

\BuiltInTok{print}\NormalTok{(}\StringTok{\textquotesingle{}Passed all tests : 10 points!\textquotesingle{}}\NormalTok{)}
\end{Highlighting}
\end{Shaded}

\begin{verbatim}
[1, 3, 6, 1, 5, 4, 1, 1, 2, 3, 3, 1, 3, 5, 2, 2, 4]
[1, 1, 1, 1, 1, 4, 5, 6, 2, 3, 3, 3, 3, 5, 2, 2, 4]
[1, 1, 1, 1, 1, 2, 2, 2, 6, 3, 3, 3, 3, 5, 5, 4, 4]
[1, 1, 1, 1, 1, 2, 2, 2, 3, 3, 3, 3, 6, 5, 5, 4, 4]
[1, 1, 1, 1, 1, 2, 2, 2, 3, 3, 3, 3, 4, 4, 5, 5, 6]
[1, 1, 1, 1, 1, 2, 2, 2, 3, 3, 3, 3, 4, 4, 5, 5, 6]
Passed all tests : 10 points!
\end{verbatim}

\hypertarget{problem-3-design-a-universal-family-hash-function}{%
\subsection{Problem 3: Design a Universal Family Hash
Function}\label{problem-3-design-a-universal-family-hash-function}}

Suppose we are interested in hashing \(n\) bit keys into \(m\) bit hash
values to hash into a table of size \(2^m\). We view our key as a bit
vector of \(n\) bits in binary. Eg., for \(n = 4\), the key
\(14 = \left(\begin{array}{c} 1\\ 1\\ 1\\ 0 \end{array} \right)\).

The hash family is defined by random boolean matrices \(H\) with \(m\)
rows and \(n\) columns. To compute the hash function, we perform a
matrix multiplication. Eg., with \(m = 3\) and \(n= 4\), we can have a
matrix \(H\) such as

\[ H = \left[ \begin{array}{cccc} 0 & 1 & 0 & 1 \\
1 & 0 & 0 & 0 \\
1 & 0 & 1 & 1 \\
\end{array} \right]\].

The value of the hash function \(H(14)\) is now obtained by multiplying

\[ \left[ \begin{array}{cccc} 0 & 1 & 0 & 1 \\
1 & 0 & 0 & 0 \\
1 & 0 & 1 & 1 \\
\end{array} \right] \times \left( \begin{array}{c}
1\\
1\\
1\\
0
\end{array} \right) \]

The matrix multiplication is carried out using AND for multiplication
and XOR instead of addition. For the example above, we compute the value
of hash function as

\[\left( \begin{array}{c}
 0 \cdot 1 + 1 \cdot 1 + 0 \cdot 1 + 1 \cdot 0 \\
 1 \cdot 1 + 0 \cdot 1 + 0 \cdot 1 + 0 \cdot 0 \\
 1 \cdot 1 + 0 \cdot 1 + 1 \cdot 1 + 1 \cdot 0 \\
 \end{array} \right) = \left( \begin{array}{c} 1 \\ 1 \\ 0 \end{array} \right)\]

(A) For a given matrix \(H\) and two keys \(x, y\) that differ only in
their \(i^{th}\) bits, provide a condition for \(Hx = Hy\) holding.
(\textbf{Hint} It may help to play with examples where you have two
numbers \(x, y\) that just differ at a particular bit position. Figure
out which entries in the matrix are multiplied with these bits that
differ).

YOUR ANSWER HERE

(B) Prove that the probability that two keys \(x, y\) such that
\(x \not= y\) collide under the random choice of a matrix \(x, y\) is at
most \(\frac{1}{2^m}\).

YOUR ANSWER HERE

\begin{Shaded}
\begin{Highlighting}[]
\ImportTok{from}\NormalTok{ random }\ImportTok{import}\NormalTok{ random}

\KeywordTok{def}\NormalTok{ dot\_product(lst\_a, lst\_b):}
\NormalTok{    and\_list }\OperatorTok{=}\NormalTok{ [elt\_a }\OperatorTok{*}\NormalTok{ elt\_b }\ControlFlowTok{for}\NormalTok{ (elt\_a, elt\_b) }\KeywordTok{in} \BuiltInTok{zip}\NormalTok{(lst\_a, lst\_b)]}
    \ControlFlowTok{return} \DecValTok{0} \ControlFlowTok{if} \BuiltInTok{sum}\NormalTok{(and\_list)}\OperatorTok{\%} \DecValTok{2} \OperatorTok{==} \DecValTok{0} \ControlFlowTok{else} \DecValTok{1}

\CommentTok{\# encode a matrix as a list of lists with each row as a list.}
\CommentTok{\# for instance, the example above is written as the matrix}
\CommentTok{\# H = [[0,1,0,1],[1,0,0,0],[1,0,1,1]]}
\CommentTok{\# encode column vectors simply as a list of elements.}
\CommentTok{\# you can use the dot\_product function provided to you.}
\KeywordTok{def}\NormalTok{ matrix\_multiplication(H, lst):}
    \CommentTok{\# your code here}
\NormalTok{    result }\OperatorTok{=}\NormalTok{ []}
    \ControlFlowTok{for}\NormalTok{ lst\_a }\KeywordTok{in}\NormalTok{ H:}
\NormalTok{      result.append(dot\_product(lst\_a, lst))}
    \ControlFlowTok{return}\NormalTok{ result}


\CommentTok{\# Generate a random m \textbackslash{}times n matrix}
\CommentTok{\# see the comment next to matrix\_multiplication for how your matrix must be returned.}
\KeywordTok{def}\NormalTok{ return\_random\_hash\_function(m, n):}
    \CommentTok{\# return a random hash function wherein each entry is chosen as 1 with probability \textgreater{}= 1/2 and 0 with probability \textless{} 1/2}
    \CommentTok{\# your code here}
    \ImportTok{import}\NormalTok{ random}
\NormalTok{    rand\_matrix }\OperatorTok{=}\NormalTok{ []}
\NormalTok{    count }\OperatorTok{=} \DecValTok{1}
    \ControlFlowTok{while}\NormalTok{ count }\OperatorTok{\textless{}=}\NormalTok{ m:}
\NormalTok{      rand\_matrix.append([random.randint(}\DecValTok{0}\NormalTok{, }\DecValTok{1}\NormalTok{) }\ControlFlowTok{for}\NormalTok{ \_ }\KeywordTok{in} \BuiltInTok{range}\NormalTok{(n)])}
\NormalTok{      count}\OperatorTok{+=}\DecValTok{1}
    \ControlFlowTok{return}\NormalTok{ rand\_matrix}
\end{Highlighting}
\end{Shaded}

\begin{Shaded}
\begin{Highlighting}[]
\NormalTok{A1 }\OperatorTok{=}\NormalTok{ [[}\DecValTok{0}\NormalTok{,}\DecValTok{1}\NormalTok{,}\DecValTok{0}\NormalTok{,}\DecValTok{1}\NormalTok{],[}\DecValTok{1}\NormalTok{,}\DecValTok{0}\NormalTok{,}\DecValTok{0}\NormalTok{,}\DecValTok{0}\NormalTok{],[}\DecValTok{1}\NormalTok{,}\DecValTok{0}\NormalTok{,}\DecValTok{1}\NormalTok{,}\DecValTok{1}\NormalTok{]]}
\NormalTok{b1 }\OperatorTok{=}\NormalTok{ [}\DecValTok{1}\NormalTok{,}\DecValTok{1}\NormalTok{,}\DecValTok{1}\NormalTok{,}\DecValTok{0}\NormalTok{]}
\NormalTok{c1 }\OperatorTok{=}\NormalTok{ matrix\_multiplication(A1, b1)}
\BuiltInTok{print}\NormalTok{(}\StringTok{\textquotesingle{}c1=\textquotesingle{}}\NormalTok{, c1)}
\ControlFlowTok{assert}\NormalTok{ c1 }\OperatorTok{==}\NormalTok{ [}\DecValTok{1}\NormalTok{,}\DecValTok{1}\NormalTok{,}\DecValTok{0}\NormalTok{] , }\StringTok{\textquotesingle{}Test 1 failed\textquotesingle{}}

\NormalTok{A2 }\OperatorTok{=}\NormalTok{ [ [}\DecValTok{1}\NormalTok{,}\DecValTok{1}\NormalTok{],[}\DecValTok{0}\NormalTok{,}\DecValTok{1}\NormalTok{]]}
\NormalTok{b2 }\OperatorTok{=}\NormalTok{ [}\DecValTok{1}\NormalTok{,}\DecValTok{0}\NormalTok{]}
\NormalTok{c2 }\OperatorTok{=}\NormalTok{ matrix\_multiplication(A2, b2)}
\BuiltInTok{print}\NormalTok{(}\StringTok{\textquotesingle{}c2=\textquotesingle{}}\NormalTok{, c2)}
\ControlFlowTok{assert}\NormalTok{ c2 }\OperatorTok{==}\NormalTok{ [}\DecValTok{1}\NormalTok{, }\DecValTok{0}\NormalTok{], }\StringTok{\textquotesingle{}Test 2 failed\textquotesingle{}}

\NormalTok{A3 }\OperatorTok{=}\NormalTok{ [ [}\DecValTok{1}\NormalTok{,}\DecValTok{1}\NormalTok{,}\DecValTok{1}\NormalTok{,}\DecValTok{0}\NormalTok{],[}\DecValTok{0}\NormalTok{,}\DecValTok{1}\NormalTok{,}\DecValTok{1}\NormalTok{,}\DecValTok{0}\NormalTok{]]}
\NormalTok{b3 }\OperatorTok{=}\NormalTok{  [}\DecValTok{1}\NormalTok{, }\DecValTok{0}\NormalTok{,}\DecValTok{0}\NormalTok{,}\DecValTok{1}\NormalTok{]}
\NormalTok{c3 }\OperatorTok{=}\NormalTok{ matrix\_multiplication(A3, b3)}
\BuiltInTok{print}\NormalTok{(}\StringTok{\textquotesingle{}c3=\textquotesingle{}}\NormalTok{, c3)}
\ControlFlowTok{assert}\NormalTok{ c3 }\OperatorTok{==}\NormalTok{ [}\DecValTok{1}\NormalTok{, }\DecValTok{0}\NormalTok{], }\StringTok{\textquotesingle{}Test 3 failed\textquotesingle{}}

\NormalTok{H }\OperatorTok{=}\NormalTok{ return\_random\_hash\_function(}\DecValTok{5}\NormalTok{,}\DecValTok{4}\NormalTok{)}
\BuiltInTok{print}\NormalTok{(}\StringTok{\textquotesingle{}H=\textquotesingle{}}\NormalTok{, H)}
\ControlFlowTok{assert} \BuiltInTok{len}\NormalTok{(H) }\OperatorTok{==} \DecValTok{5}\NormalTok{, }\StringTok{\textquotesingle{}Test 5 failed\textquotesingle{}}
\ControlFlowTok{assert} \BuiltInTok{all}\NormalTok{(}\BuiltInTok{len}\NormalTok{(row) }\OperatorTok{==} \DecValTok{4} \ControlFlowTok{for}\NormalTok{ row }\KeywordTok{in}\NormalTok{ H), }\StringTok{\textquotesingle{}Test 6 failed\textquotesingle{}}
\ControlFlowTok{assert} \BuiltInTok{all}\NormalTok{(elt }\OperatorTok{==} \DecValTok{0} \KeywordTok{or}\NormalTok{ elt }\OperatorTok{==} \DecValTok{1} \ControlFlowTok{for}\NormalTok{ row }\KeywordTok{in}\NormalTok{ H }\ControlFlowTok{for}\NormalTok{ elt }\KeywordTok{in}\NormalTok{ row ),  }\StringTok{\textquotesingle{}Test 7 failed\textquotesingle{}}

\NormalTok{H2 }\OperatorTok{=}\NormalTok{ return\_random\_hash\_function(}\DecValTok{6}\NormalTok{,}\DecValTok{3}\NormalTok{)}
\BuiltInTok{print}\NormalTok{(}\StringTok{\textquotesingle{}H2=\textquotesingle{}}\NormalTok{, H2)}
\ControlFlowTok{assert} \BuiltInTok{len}\NormalTok{(H2) }\OperatorTok{==} \DecValTok{6}\NormalTok{, }\StringTok{\textquotesingle{}Test 8 failed\textquotesingle{}}
\ControlFlowTok{assert} \BuiltInTok{all}\NormalTok{(}\BuiltInTok{len}\NormalTok{(row) }\OperatorTok{==} \DecValTok{3} \ControlFlowTok{for}\NormalTok{ row }\KeywordTok{in}\NormalTok{ H2),  }\StringTok{\textquotesingle{}Test 9 failed\textquotesingle{}}
\ControlFlowTok{assert} \BuiltInTok{all}\NormalTok{(elt }\OperatorTok{==} \DecValTok{0} \KeywordTok{or}\NormalTok{ elt }\OperatorTok{==} \DecValTok{1} \ControlFlowTok{for}\NormalTok{ row }\KeywordTok{in}\NormalTok{ H2 }\ControlFlowTok{for}\NormalTok{ elt }\KeywordTok{in}\NormalTok{ row ), }\StringTok{\textquotesingle{}Test 10 failed\textquotesingle{}}
\BuiltInTok{print}\NormalTok{(}\StringTok{\textquotesingle{}Tests passed: 10 points!\textquotesingle{}}\NormalTok{)}
\end{Highlighting}
\end{Shaded}

\begin{verbatim}
c1= [1, 1, 0]
c2= [1, 0]
c3= [1, 0]
H= [[0, 1, 0, 1], [1, 1, 0, 0], [1, 0, 0, 1], [0, 1, 1, 1], [1, 0, 1, 1]]
H2= [[0, 0, 0], [1, 1, 0], [0, 0, 0], [1, 1, 1], [1, 0, 1], [1, 1, 1]]
Tests passed: 10 points!
\end{verbatim}

\hypertarget{manually-graded-answers}{%
\subsection{Manually Graded Answers}\label{manually-graded-answers}}

\hypertarget{problem-1}{%
\subsubsection{Problem 1}\label{problem-1}}

The bug is in the initialization of i in the algorithm. It must be i =-1
rather than i = 0. Due to this, either the first element of the array is
never considered during the partition or there could be an access to i+1
that is out of array bounds.


\hypertarget{problem-2-a}{%
\subsubsection{Problem 2 A}\label{problem-2-a}}

\begin{verbatim}
for k = 1 to n
   j = partition array a with k as pivot
\end{verbatim}

The running time is \(\Theta(n \times k)\).

\hypertarget{problem-3-a}{%
\subsubsection{Problem 3 A}\label{problem-3-a}}

Since \(x,y\) differe only in their \(i^{th}\) bits, we can assume
\(x_i = 0\) and \(y_i = 1\). Therefore, \(H x + H_i = Hy\) wherein,
\(+\) refers to entrywise XOR and \(H_i\) is the \(i^{th}\) column of
\(H\). Thus, \(Hx = Hy\) if and only if \(H_i\) has all zeros. This
happens with probability \(\frac{1}{2^m}\).

\hypertarget{problem-3-b}{%
\subsubsection{Problem 3 B}\label{problem-3-b}}

Let us assume that \(x\) and \(y\) differ in \(k\) out of \(n\)
positions. We know that \(Hx = Hy\) if and only if \(Hx + Hy = 0\) where
\(+\) is XOR and \(0\) is the vector of all zeros. But
\(Hx + Hy = H (x + y)\) since AND distributes over XOR.

Whenever \(x\) and \(y\) agree in the \(i^{th}\) entries, we have the
\(i^{th}\) entry of \((x+y)\) is zero. Therefore, \(H(x+y)\) is just the
XOR sum of \(k\) columns of \(H\) corresponding to positions where \(x\)
and \(y\) differ.

Thus, one of the columns must equal the sum of the remaining \(k-1\)
columns. Let us fix these \(k-1\) columns as given and the last column
as randomly chosen. The probability that each of the \(m\) entries of
the last column matches the sum of the first \(k-1\) column is
\(\frac{1}{2^m}\).

\hypertarget{thats-all-folks}{%
\subsection{That\textquotesingle s all folks}\label{thats-all-folks}}

\end{document}
